\documentclass[10pt]{article}
\usepackage[utf8]{inputenc}
\usepackage[dvips]{graphicx}
\usepackage{fancybox}
\usepackage{verbatim}
\usepackage{array}
\usepackage{latexsym}
\usepackage{alltt}
\usepackage{hyperref}
\usepackage{textcomp}
\usepackage{color}
\usepackage{amsmath}
\usepackage{amsfonts}
\usepackage{tikz}
\usepackage{fitch}  % to use fitch
\usepackage{float}
\usepackage[hmargin=3cm,vmargin=5.0cm]{geometry}
%\topmargin=0cm
\topmargin=-2cm
\addtolength{\textheight}{6.5cm}
\addtolength{\textwidth}{2.0cm}
%\setlength{\leftmargin}{-5cm}
\setlength{\oddsidemargin}{0.0cm}
\setlength{\evensidemargin}{0.0cm}


\begin{document}

\section*{Student Information } 
%Write your full name and id number between the colon and newline
%Put one empty space character after colon and before newline
Full Name :  Emre Geçit \\
Id Number :  2521581 \\

\section*{Q. 1}
Assume that there exists some a such that $a < 1$.\\
$a < 1\\
a * a < a\\
a\textsuperscript{2} < a$\\Since a is an element of positive integers, $a\textsuperscript{2}$ must also be an element of positive integers. Since, in the beginning we have assumed that a is the smallest positive integer, and we reached a positive integer that is smaller than a, we reached a contradiction. Therefore, there cannot be a positive integer that is smaller than 1. By the well-ordering property of the positive integers, we can conclude that 1 is the smallest positive integer.
\section*{Q. 2}
Proving S(m, 1):\\Basis step:\\S(1, 1):\\$x\textsubscript{1}=1$, There is one solution to this problem, $x\textsubscript{1}=1.\\f(1, 1) = \dfrac{1!}{1*1!} = 1$\\\\
Induction: Assume that S(j, 1) holds.\\ for S(j+1, 1):\\
$x\textsubscript{1}+...+x\textsubscript{j}+x\textsubscript{j+1} = 1$\\
There are two different solutions to this problem:\\ $x\textsubscript{1}+...+x\textsubscript{j}=1, x\textsubscript{j+1}=0$ or $x\textsubscript{1}+...+x\textsubscript{j}=0, x\textsubscript{j+1}=1$\\ $x\textsubscript{1}+...+x\textsubscript{j}=1$ has f(j, 1) = $\dfrac{j!}{(j-1)!} = j$ solutions (by the assumption).\\ At the end, the equation $x\textsubscript{1}+...+x\textsubscript{j}+x\textsubscript{j+1} = 1$ has j+1 solutions.\\ f(j+1, 1) = $\dfrac{(j+1)!}{j!} = j+1$ holds.\\
Conclusion:\\
S(m, 1) proved using induction.\\\\Proving S(1, n):\\$x\textsubscript{1}=j$\\This kind of an equation always has one solution (independent of j).\\ f(1, j) = $\dfrac{j!}{j!*0!}$ = 1.
\\ S(m, 1) proved.\\\\
Proving $f(m+1,n+1)=f(m+1,n)+f(m,n+1)$:\\
S(m+1, n+1) states that $x\textsubscript{1}+...+x\textsubscript{m}+x\textsubscript{m+1}=n+1\\\\
x\textsubscript{m+1}==0 \rightarrow x\textsubscript{1}+...+x\textsubscript{m}=n$ (f(m,n) different solutions)\\$x\textsubscript{m+1}==1 \rightarrow x\textsubscript{1}+...+x\textsubscript{m}=n-1$ (f(m,n-1) different solutions)\\......\\$x\textsubscript{m+1}==n \rightarrow x\textsubscript{1}+...+x\textsubscript{m}=0$ (1 solution)\\\\
In total there are $\sum_{n=1} ^{n} f(m, i) + 1$ solutions.\\\\
f(m+1, n+1) = $\sum_{i=1} ^{n} f(m, i) + 1\\=f(m,n)+\sum_{i=1} ^{n-1} f(m, i)+1\\=f(m,n)+f(m+1, n-1)$\\\\Conclusion:\\$f(m+1,n+1)=f(m+1,n)+f(m,n+1)$ proved.\\Proving $f(m+1, n+1) = \dfrac{(n+m)!}{(n+1)!m!}$:\\\\
Basis step:\\
S(m,1) proved before.\\
S(1,n) proved before.\\\\
Induction:\\
Assume that S(j+1,k) and S(j,k+1) holds.\\\\
Then:\\$
f(j+1, k) = \dfrac{(j+k)!}{k!*j!}\\\\
f(j, k+1) = \dfrac{(j+k)!}{(k+1)!*(j-1)!}\\\\
f(j+1, k+1) = f(j+1, k) + f(j, k+1)\\\\
= \dfrac{(j+k)!}{k!*j!} + \dfrac{(j+k)!}{(k+1)!*(j-1)!}\\\\
= \dfrac{(k+1)*(j+k)!}{(k+1)!*j!} + \dfrac{j*(j+k)!}{(k+1)!*j!}\\\\
= \dfrac{(k+1+j)*(j+k)!}{(k+1)!*j!}\\\\
= \dfrac{(k+1+j)!}{(k+1)!*j!}\\\\
$
Conclusion:
If S(j+1,k) and S(j,k+1) are true, then S(j+1,k+1) is also true. Since the base cases are also true, we can conclude S(m,n) by induction.

\section*{Q. 3}
a.\\\\
There can be 4 orientation for a triangle based on the orientation of its hypotenuse.\\For the up-right oriented triangle, there can be 1+2+3+4+5+6+7 = 7*(7+1)/2 = 28 triangles.\\For the up-left oriented triangle there can be 1+2+3+4+5+6 = 6*(6+1) = 21 triangles.\\For the down-left oriented triangle there can be 1+2+3+4+5+6 = 6*(6+1) = 21 triangles.\\For the down-right oriented triangle there can be 1+2+3+4+5+6 = 6*(6+1) = 21 triangles.\\
Answer: 21+21+21+28 = 91\\\\b.\\At least one value in the image should be mapped by more than one values in the domain. There are two possibilities:\\1) A value in the image will be mapped by three values, and the rest will be one-to-one (3, 1, 1, 1).\\2) 2 different values in the image will be mapped by two values, and the rest will be one-to-one (2, 2, 1, 1).\\\\ The number of occurences of the possibility 1 is: C(6, 3)*C(3, 1)*C(2, 1)*C(1, 1)*4 = 480.\\ The number of occurences of the possibility 2 is: C(6, 2)*C(4, 2)*C(2, 1)*C(1, 1)*6 = 1080.\\Total sum: 1080 + 480 = 1560

\section*{Q. 4}
a.
For any string of length n, $n>1$, if the first n-1 character constitutes a valid string, there are 3 possible states that the last character can take. If, first n-1 character does not constitute a valid string, then there is only one state that the last character can take. Therefore,\\
$a\textsubscript{n} = 2a\textsubscript{n-1} + 3\textsuperscript{n-1}$\\\\b.\\$a\textsubscript{1} = 0$\\\\c.\\Homogenous solution:\\$a\textsubscript{n}-2a\textsubscript{n-1}=0$\\\\Characteristic equation:\\$r-2=0\\r=2\\a\textsubscript{n}\textsuperscript{h}=A*2\textsuperscript{n}$\\\\Particular solution:\\
$a\textsubscript{n}=A*3\textsuperscript{n}\\A*3\textsuperscript{n}=2A*3\textsuperscript{n-1}+3\textsuperscript{n-1}\\3*A*3\textsuperscript{n-1}=2A*3\textsuperscript{n-1}+3\textsuperscript{n-1}\\A*3\textsuperscript{n-1}=3\textsuperscript{n-1}\\A=1\\a\textsubscript{n}\textsuperscript{h}=3\textsuperscript{n}$\\\\
General Solution:\\
$a\textsubscript{n}=a\textsubscript{n}\textsuperscript{h}+a\textsubscript{n}\textsuperscript{p}=A*2\textsuperscript{n}+3\textsuperscript{n}\\
a\textsubscript{1}=A*2\textsuperscript{1}+3\textsuperscript{1}=0\\
A=-3/2\\
a\textsubscript{n}=-3*2\textsuperscript{n-1}+3\textsuperscript{n}$

\end{document}